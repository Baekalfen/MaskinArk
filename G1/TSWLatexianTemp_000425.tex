\documentclass[a4,11pt]{article} 
\usepackage[T1]{fontenc}
\usepackage[utf8]{inputenc}
\usepackage{geometry}
\geometry{letterpaper}
\usepackage{graphicx}
\usepackage[danish]{babel}
\usepackage{amssymb}
\usepackage{fancyhdr}
\usepackage{amsmath,amssymb}
\usepackage{comment}
\usepackage{caption}
\usepackage{subfigure}
\usepackage{fixltx2e}
\usepackage{changepage}
\usepackage{listings}
\DeclareUnicodeCharacter{00A0}{ }
\pagestyle{fancy}

% http://tex.stackexchange.com/questions/33519/vertical-line-in-matrix-using-latexit
% Fix for bmatrix
\makeatletter
\renewcommand*\env@matrix[1][*\c@MaxMatrixCols c]{%
  \hskip -\arraycolsep
  \let\@ifnextchar\new@ifnextchar
  \array{#1}}
\makeatother

% My commands
\newcommand{\equ}[1]{\begin{align}#1\end{align}}
\newcommand{\tb}[1]{\textbf{#1}\\}
\newcommand{\tsub}[1]{\textsubscript{#1}}
\newcommand{\tsup}[1]{\textsuperscript{#1}}

% End of my commands

\begin{document}
\section{feb2013 - Opgave 1}
\subsection{}
\textbf{Konverter følgende decimaltal til binære tal; angiv svaret i 8-bit 2-komplementform: 0, 1, -1, 200, -100.}
\\\\
Konverteringerne kan ses i tabellen nedenfor hvor decimaltallet 200 giver et overflow.\\
Måden vi har konverteret på er ved at sætte 1 i cellerne således at summen af de steder hvor der er 1 giver decimaltallet for hver række. 
\begin{center}
\begin{tabular}{| r | c | c | c | c | c | c | c | c |}
\hline
   		& Sign 	& 64 & 32 & 16 & 8 & 4 & 2 & 1\\\hline
  0 		& 0 		& 0 & 0 & 0 & 0 & 0 & 0 & 0 \\\hline
  1 		& 0 		& 0 & 0 & 0 & 0 & 0 & 0 & 1\\\hline
  -1 		& 1 		& 1 & 1 & 1 & 1 & 1 & 1 & 1 \\\hline
  200		& - 		& - & - & - & - & - & - & -\\\hline
  -100 	& 1 		& 0 & 0 & 1 & 1 & 1 & 0 & 1\\ \hline
\end{tabular}
\end{center}


\subsection{}
\textbf{Konverter følgende binære tal angivet i 8-bit 2-komplementform til decimal-tal: 00011000, 01110000, 10000000, 11111111, 10101010.}
\\\\
Konverteringerne kan ses i tabellen nedenfor hvor der er regnet omvendt i forhold til opgaven ovenfor.
\begin{center}
\begin{tabular}{| r | c | c | c | c | c | c | c | r |}
\hline
Sign   	& 64 	& 32 & 16 & 8 & 4 & 2 & 1 & Resultat\\\hline
0 		& 0 		& 0 & 1 & 1 & 0 & 0 & 0 & 24 \\\hline
0 		& 1 		& 1 & 1 & 0 & 0 & 0 & 0 & 112\\\hline
1 		& 0 		& 0 & 0 & 0 & 0 & 0 & 0 & -127 \\\hline
1		& 1 		& 1 & 1 & 1 & 1 & 1 & 1 & -1\\\hline
1 		& 0 		& 1 & 0 & 1 & 0 & 1 & 0 & 42\\ \hline
\end{tabular}
\end{center}

\subsection{}
\textbf{Hvilke af følgende beregninger vil give overløb i 8-bit 2-komplementformat: 126+1, 127+2, -128+1, -12*12,-11*(-11).}
\\\\
Da det største tal man kan repræsenterer med 7-bit(da vi bruger 1-bit til fortegn) er:
$$1+2+4+8+16+32+64=128$$
Vil den eneste beregning der giver overløb være 127+2.\\
\\
-128+1 kan godt lade sig gøre da det mindste tal vi kan repræsenterer er -127.

\end{document}