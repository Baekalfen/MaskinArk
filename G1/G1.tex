\documentclass[a4paper,11pt]{article}

\usepackage[T1]{fontenc}
\usepackage[utf8]{inputenc}
\usepackage{geometry}
\geometry{letterpaper}
\usepackage{graphicx}
\usepackage[danish]{babel}
\usepackage{fancyhdr}
\usepackage{comment}
\usepackage{caption}
\usepackage{subfigure}
\usepackage{fixltx2e}
\usepackage{changepage}
\usepackage{listings}
\DeclareUnicodeCharacter{00A0}{ }
\pagestyle{fancy}

% My commands
\newcommand{\tb}[1]{\textbf{#1}\\}
\newcommand{\tsub}[1]{\textsubscript{#1}}
\newcommand{\tsup}[1]{\textsuperscript{#1}}

% End of my commands








% --------------------------- MADS MANGLER PAKKER --------------------------------
















\newcommand{\Fag}{Matematik}
\newcommand{\Navn}{Mads~Ynddal}
\newcommand{\Skole}{Roskilde~Tekniske~Gymnasium}
\newcommand{\Klasse}{3.4}
\newcommand{\Titel}{Distributed~Computing}
\newcommand{\Vejleder}{J\o rn~Bendtsen}

\date{\Large{19. marts 2013}}

\author{\huge{\Navn}\\\Skole\\Klasse \Klasse\\\Fag\\\\Vejleder: \Vejleder}
\title{\Huge{\Titel}}
\usepackage[utf8]{inputenc} 

\newcommand{\degree}{\ensuremath{^\circ}}
\usepackage[british,danish]{babel}
\usepackage{amssymb}
\usepackage{graphicx}
\usepackage{calc}

\newcommand{\pic}[4] {\begin{figure}[ht!]\centering\includegraphics[width=#2]{#1}\caption{#3}\label{#4}\end{figure}}
\newcommand{\vektor}[1]{\overrightarrow{#1}}
\newcommand{\matriks}[1]{\left(\begin{matrix}#1\end{matrix}\right)}
\newcommand{\equ}[1]{\begin{align}#1\end{align}}
\newcommand{\fodurl}[3]{\footnote{\textbf{Ophav:} #1, \textbf{URL}: #2, \textbf{Dato:} #3}}
\newcommand{\fodbog}[5]{\footnote{\textbf{Forfatter:} #1, \textbf{Titel:} #2, \textbf{Udgiver:} #3, \textbf{År:} #4, \textbf{Side:} #5}}
\newcommand{\listeurl}[3]{\item \textbf{Ophav:} #1, \textbf{URL}: #2, \textbf{Dato:} #3}
\newcommand{\listebog}[5]{\item \textbf{Forfatter:} #1, \textbf{Titel:} #2, \textbf{Udgiver:} #3, \textbf{År:} #4, \textbf{Side:} #5}
\newcommand{\definition}[2]{\begin{center}\fbox{\parbox{\textwidth-5cm}{\begin{center}\textbf{#1}\\\emph{#2}\end{center}}}\end{center}}
\newcommand{\liste}[1]{\begin{itemize}#1\end{itemize}}

\addtolength{\textwidth}{100pt}
\addtolength{\hoffset}{-50pt}

\usepackage{fancyhdr}
\pagestyle{fancy}
% with this we ensure that the chapter and section
% headings are in lowercase.
\renewcommand{\sectionmark}[1]{%
        \markright{\thesection\ #1}}
\fancyhf{}  % delete current header and footer
%\fancyhead[LE,RO]{\bfseries\thepage}
\fancyhead[RO]{\bfseries Side: \thepage}
\fancyhead[LO]{\bfseries  Asger Lund Hansen, Mads Ynddal og Troels Ynddal}
%\fancyhead[RE]{\bfseries\leftmark}
\renewcommand{\headrulewidth}{0.5pt}
\renewcommand{\footrulewidth}{0pt}
\addtolength{\headheight}{0.5pt} % space for the rule
\fancypagestyle{plain}{%
   \fancyhead{} % get rid of headers on plain pages
   \renewcommand{\headrulewidth}{0pt} % and the line
}

\usepackage{listings} %Code highlighting

\begin{document}
\frenchspacing
%\selectlanguage{danish}

%\maketitle
\newpage
\section*{Fibonacci i MIPS}
Følgende MIPS kode vil udregne Fibonaccitallet til 10, eller et andet tal, ved at ændre det på næstsidste linje. Kommentarer i koden beskriver hvert trin.
\begin{verbatim}
    j    main

fib:    ble  $a0,1,returnn      # if n<=1, jump to returnn label

    # save return address
    addi $sp,$sp,-8         # make room in stack for ra and n
    sw   $ra,4($sp)         # save ra in stack

    # fib(n-1)
    sw   $a0,0($sp)         # save n in stack because it gets overwritten in the recursive call
    addi $a0,$a0,-1         # decrement n by 1
    jal  fib#(n-1)          # jump to fib(n-1)

    # fib(n-2)
    lw   $a0,0($sp)         # load old n (before fib(n-1))
    addi $sp,$sp,4          # close used space in stack
    addi $a0,$a0,-2         # decrement n by 2 for n-2
    jal  fib#(n-2)          # jump to fib(n-2)

    # return
    lw   $ra,0($sp)         # load old ra before return
    addi $sp,$sp,4          # close used space in stack
    jr   $ra                # jump back to calling address


returnn:add  $v0,$v0,$a0        # add n to $v0 (where the result is stored)
    jr   $ra                # jump back to function call


main:   addi $a0,$zero,10   # init n as a0 to its start value
    jal  fib                # jump to the fib function (label)
\end{verbatim}

\section*{Eksamenssæt 2012}
% \tb{}
\subsection*{a)}

Konverter følgende fem decimaltal til hexadecimaltal: 0, 13, 314, 1337, 7913. Giv alle svarene på tre cifre.

Det kræver ingen udregninger at konvertere $0$ og $13$ til hexadecimal, da hexadecimals første ciffer går op til 16. Vi kan derfor blot skrive: 
$$0_{10} = 0_{16}$$
$$13_{10} = D_{16}$$
Dette er dog ikke tilfældet med $314_{10}$, som kræver en regning. Vi dividere 314 med 16 og dividere resultatet af det med 16 igen, indtil at delresultatet bliver mindre end 16.
$$314_{10}/16 = 19 + \frac{10}{16}$$
$$19/16 = 1 + \frac{3}{16}$$
$$1/16 = \frac{1}{16}$$

Da $1 < 16$, kan vi ikke regne videre, og må derfor aflæse resultatet. Tælleren af den første brøk er vores ``1'ere'', den næste brøks tæller er vores ``16'ere'' osv. Vi får derfor:
$$314_{10} = 13D_{16}$$

Omregningen af $1337_{10}$ foregår på samme måde:
$$1337_{10}/16 = 83 + \frac{9}{16}$$
$$83_{10}/16 = 5 + \frac{3}{16}$$
$$5_{10}/16 = \frac{5}{16}$$
$$1337_{10} = 539_{16}$$

Det sidste tal er $7913_{10}$, men da det største tal vi kan repræsentere med hexadecimal er $256*16+16*16+1*16 = 4.368_{10}$, kan vi ikke omregne $7913_{10}$.

\subsection*{b)}
\tb{}

Konverter følgende hexadecimaltal til decimaltal: 800, F2F, FFF, 10, DAD. 

(Tallene er i base 10, med mindre andet er angivet.)
$$ 800_{16} = 8*256 + 0 * 16 + 0 * 1 = 2048_{10}$$
$$ F2F_{16} = 15 * 256 + 2 * 16 + 15 * 1 = 3887_{10}$$
$$ FFF_{16} = 15 * 256 + 15 * 16 + 15 * 1 = 4095_{10}$$
$$ 10_{16} = 0* 256 + 1 * 16 + 0 * 1 = 16_{10}$$
$$ DAD_{16} = 14 * 256 + 10 * 16 + 14 * 1 = 3758_{10}$$

\subsection*{c)}
\tb{}

\begin{center}
  \begin{tabular}{| c | c | c | c |}
    \hline
     \textbf{4096} & \textbf{256} & \textbf{16} & \textbf{1}\\ \hline
    1 & E & E & 9\\
    \hline
  \end{tabular}
\end{center}

Da det er nemmere at omregne hexadecimal til binæretal, bruger vi tallene fra opgave 1.1\\
Vi opstiller følgende tabel, så vi nemt og hurtigt kan omregne til binære tal. Vi starter med at ignorer fortegnet på talene.
\begin{center}
  \begin{tabular}{ | l | c | c | c |  c |  c |  c |  c |  c |  c |  c |  c |  c |  c | }
    \hline
    & Sign & 2048 & 1024 & 512 & 256 & 128 & 64 & 32 & 16 & 8 & 4 & 2 & 1\\ \hline
    $0_{16}$ &        0 & 0 & 0 & 0 & 0 & 0 & 0 & 0 & 0 & 0 & 0 & 0 & 0 \\ \hline
    $D_{16}$ &      0 & 0 & 0 & 0 & 0 & 0 & 0 & 0 & 0 & 1 & 1 & 0 & 1 \\ \hline
    $13D_{16}$ &    0&0&0&0&1&0&0&1&1&1&1&1&0 \\ \hline
    $539_{16}$ &   0 & 0 & 1 & 0 & 1 & 0 & 0 & 0 & 0 & 0 & 0 & 0 & 0 \\ \hline
    $1DD9_{16}$ & 1 & 1 & 1 & 1 & 1 & 0 & 0 & 1 & 0 & 1 & 0 & 0 & 1 \\
    \hline
  \end{tabular}
\end{center}

Vi ser, at  $1DD9_{16}$ er for stort til at blive repræsenteret med 12 bit. Vi har derfor et overflow, når vi omregner  $1DD9_{16}$.\\

Vi ser bort fra $-1DD9_{16}$ og fortsætter med at konvertere til 12 bit 2-komplement
\begin{center}
  \begin{tabular}{ | l | c | c | c |  c |  c |  c |  c |  c |  c |  c |  c |  c |  c | }
   \hline
    & Sign & 2048 & 1024 & 512 & 256 & 128 & 64 & 32 & 16 & 8 & 4 & 2 & 1\\ \hline
    $0_{16}$ &         0 & 0 & 0 & 0 & 0 & 0 & 0 & 0 & 0 & 0 & 0 & 0 & 0 \\ \hline
    $-D_{16}$ &     1 & 1 & 1 & 1 & 1 & 1 & 1 & 1 & 1 & 0 & 0 & 1 & 0 \\ \hline
    $13D_{16}$ &   1&1&1&1&0&1&1&0&0&0&0&1&0 \\ \hline
    $-539_{16}$ &   1 & 1 & 0 & 1 & 1 & 0 & 0 & 0 & 0 & 0 & 0 & 0 & 0 \\ \hline
    $-1DD9_{16}$ & 1 & - & - & - & - & - & - & - & - & - & - & - & - \\ \hline
  \end{tabular}
\end{center}

\section*{Eksamenssæt 2013}
\textbf{Konverter følgende decimaltal til binære tal; angiv svaret i 8-bit 2-komplementform: 0, 1, -1, 200, -100.}
\\\\
Konverteringerne kan ses i tabellen nedenfor hvor decimaltallet 200 giver et overflow.\\
Måden vi har konverteret på er ved at sætte 1 i cellerne således at summen af de steder hvor der er 1 giver decimaltallet for hver række. 
\begin{center}
\begin{tabular}{| r | c | c | c | c | c | c | c | c |}
\hline
        & Sign  & 64 & 32 & 16 & 8 & 4 & 2 & 1\\\hline
  0         & 0         & 0 & 0 & 0 & 0 & 0 & 0 & 0 \\\hline
  1         & 0         & 0 & 0 & 0 & 0 & 0 & 0 & 1\\\hline
  -1        & 1         & 1 & 1 & 1 & 1 & 1 & 1 & 1 \\\hline
  200       & -         & - & - & - & - & - & - & -\\\hline
  -100  & 1         & 0 & 0 & 1 & 1 & 1 & 0 & 1\\ \hline
\end{tabular}
\end{center}


\subsection*{}
\textbf{Konverter følgende binære tal angivet i 8-bit 2-komplementform til decimal-tal: 00011000, 01110000, 10000000, 11111111, 10101010.}
\\\\
Konverteringerne kan ses i tabellen nedenfor hvor der er regnet omvendt i forhold til opgaven ovenfor.
\begin{center}
\begin{tabular}{| r | c | c | c | c | c | c | c | r |}
\hline
Sign    & 64    & 32 & 16 & 8 & 4 & 2 & 1 & Resultat\\\hline
0       & 0         & 0 & 1 & 1 & 0 & 0 & 0 & 24 \\\hline
0       & 1         & 1 & 1 & 0 & 0 & 0 & 0 & 112\\\hline
1       & 0         & 0 & 0 & 0 & 0 & 0 & 0 & -127 \\\hline
1       & 1         & 1 & 1 & 1 & 1 & 1 & 1 & -1\\\hline
1       & 0         & 1 & 0 & 1 & 0 & 1 & 0 & 42\\ \hline
\end{tabular}
\end{center}

\subsection*{}
\textbf{Hvilke af følgende beregninger vil give overløb i 8-bit 2-komplementformat: 126+1, 127+2, -128+1, -12*12,-11*(-11).}
\\\\
Da det største tal man kan repræsenterer med 7-bit(da vi bruger 1-bit til fortegn) er:
$$1+2+4+8+16+32+64=128$$
Vil den eneste beregning der giver overløb være 127+2.\\
\\
-128+1 kan godt lade sig gøre da det mindste tal vi kan repræsenterer er -127.

\end{document}