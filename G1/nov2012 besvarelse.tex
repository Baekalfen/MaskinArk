\documentclass[a4,11pt]{article} 
\usepackage[T1]{fontenc}
\usepackage[utf8]{inputenc}
\usepackage{geometry}
\geometry{letterpaper}
\usepackage{graphicx}
\usepackage[danish]{babel}
\usepackage{amssymb}
\usepackage{fancyhdr}
\usepackage{amsmath,amssymb}
\usepackage{comment}
\usepackage{caption}
\usepackage{subfigure}
\usepackage{fixltx2e}
\usepackage{changepage}
\usepackage{listings}
\DeclareUnicodeCharacter{00A0}{ }
\pagestyle{fancy}

% http://tex.stackexchange.com/questions/33519/vertical-line-in-matrix-using-latexit
% Fix for bmatrix
\makeatletter
\renewcommand*\env@matrix[1][*\c@MaxMatrixCols c]{%
  \hskip -\arraycolsep
  \let\@ifnextchar\new@ifnextchar
  \array{#1}}
\makeatother

% My commands
\newcommand{\equ}[1]{\begin{align}#1\end{align}}
\newcommand{\tb}[1]{\textbf{#1}\\}
\newcommand{\tsub}[1]{\textsubscript{#1}}
\newcommand{\tsup}[1]{\textsuperscript{#1}}

% End of my commands

\begin{document}
\section{Opgave}

\subsection{a)}
\tb{}

Konverter følgende fem decimaltal til hexadecimaltal: 0, 13, 314, 1337, 7913. Giv alle svarene på tre cifre.

Det kræver ingen udregninger at konvertere $0$ og $13$ til hexadecimal, da hexadecimals første ciffer går op til 16. Vi kan derfor blot skrive: 
$$0_{10} = 0_{16}$$
$$13_{10} = D_{16}$$
Dette er dog ikke tilfældet med $314_{10}$, som kræver en regning. Vi dividere 314 med 16 og dividere resultatet af det med 16 igen, indtil at delresultatet bliver mindre end 16.
$$314_{10}/16 = 19 + \frac{10}{16}$$
$$19/16 = 1 + \frac{3}{16}$$
$$1/16 = \frac{1}{16}$$

Da $1 < 16$, kan vi ikke regne videre, og må derfor aflæse resultatet. Tælleren af den første brøk er vores ``1'ere'', den næste brøks tæller er vores ``16'ere'' osv. Vi får derfor:
$$314_{10} = 13D_{16}$$

Omregningen af $1337_{10}$ foregår på samme måde:
$$1337_{10}/16 = 83 + \frac{9}{16}$$
$$83_{10}/16 = 5 + \frac{3}{16}$$
$$5_{10}/16 = \frac{5}{16}$$
$$1337_{10} = 539_{16}$$

Det sidste tal er $7913_{10}$, men da det største tal vi kan repræsentere med hexadecimal er $256*16+16*16+1*16 = 4.368_{10}$, kan vi ikke omregne $7913_{10}$.

\subsection{b)}
\tb{}

Konverter følgende hexadecimaltal til decimaltal: 800, F2F, FFF, 10, DAD. 

(Tallene er i base 10, med mindre andet er angivet.)
$$ 800_{16} = 8*256 + 0 * 16 + 0 * 1 = 2048_{10}$$
$$ F2F_{16} = 15 * 256 + 2 * 16 + 15 * 1 = 3887_{10}$$
$$ FFF_{16} = 15 * 256 + 15 * 16 + 15 * 1 = 4095_{10}$$
$$ 10_{16} = 0* 256 + 1 * 16 + 0 * 1 = 16_{10}$$
$$ DAD_{16} = 14 * 256 + 10 * 16 + 14 * 1 = 3758_{10}$$

\subsection{c)}
\tb{}

\begin{center}
  \begin{tabular}{| c | c | c | c |}
    \hline
     \textbf{4096} & \textbf{256} & \textbf{16} & \textbf{1}\\ \hline
    1 & E & E & 9\\
    \hline
  \end{tabular}
\end{center}

Da det er nemmere at omregne hexadecimal til binæretal, bruger vi tallene fra opgave 1.1\\
Vi opstiller følgende tabel, så vi nemt og hurtigt kan omregne til binære tal. Vi starter med at ignorer fortegnet på talene.
\begin{center}
  \begin{tabular}{ | l | c | c | c |  c |  c |  c |  c |  c |  c |  c |  c |  c |  c | }
    \hline
    & Sign & 2048 & 1024 & 512 & 256 & 128 & 64 & 32 & 16 & 8 & 4 & 2 & 1\\ \hline
    $0_{16}$ &        0 & 0 & 0 & 0 & 0 & 0 & 0 & 0 & 0 & 0 & 0 & 0 & 0 \\ \hline
    $D_{16}$ &      0 & 0 & 0 & 0 & 0 & 0 & 0 & 0 & 0 & 1 & 1 & 0 & 1 \\ \hline
    $FORKERT$ &   0 & 0 & 0 & 1 & 1 & 0 & 0 & 0 & 1 & 0 & 1 & 0 & 0 \\ \hline
    $539_{16}$ &   0 & 0 & 1 & 0 & 1 & 0 & 0 & 0 & 0 & 0 & 0 & 0 & 0 \\ \hline
    $1DD9_{16}$ & 1 & 1 & 1 & 1 & 1 & 0 & 0 & 1 & 0 & 1 & 0 & 0 & 1 \\
    \hline
  \end{tabular}
\end{center}

Vi ser, at  $1DD9_{16}$ er for stort til at blive repræsenteret med 12 bit. Vi har derfor et overflow, når vi omregner  $1DD9_{16}$.\\

Vi ser bort fra $-1DD9_{16}$ og fortsætter med at konvertere til 12 bit 2-komplement
\begin{center}
  \begin{tabular}{ | l | c | c | c |  c |  c |  c |  c |  c |  c |  c |  c |  c |  c | }
   \hline
    FORKERT
    & Sign & 2048 & 1024 & 512 & 256 & 128 & 64 & 32 & 16 & 8 & 4 & 2 & 1\\ \hline
    $0_{16}$ &         0 & 0 & 0 & 0 & 0 & 0 & 0 & 0 & 0 & 0 & 0 & 0 & 0 \\ \hline
    $-D_{16}$ &     1 & 1 & 1 & 1 & 1 & 1 & 1 & 1 & 1 & 0 & 0 & 1 & 0 \\ \hline
    $FORKERT$ &   1 & 1 & 1 & 0 & 0 & 1 & 1 & 1 & 0 & 1 & 1 & 0 & 0 \\ \hline
    $-539_{16}$ &   1 & 1 & 0 & 1 & 1 & 0 & 0 & 0 & 0 & 0 & 0 & 0 & 0 \\ \hline
    $-1DD9_{16}$ & 1 & - & - & - & - & - & - & - & - & - & - & - & - \\ \hline
  \end{tabular}
\end{center}























\end{document}