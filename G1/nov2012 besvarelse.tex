\documentclass[a4,11pt]{article} 
\usepackage[T1]{fontenc}
\usepackage[utf8]{inputenc}
\usepackage{geometry}
\geometry{letterpaper}
\usepackage{graphicx}
\usepackage[danish]{babel}
\usepackage{amssymb}
\usepackage{fancyhdr}
\usepackage{amsmath,amssymb}
\usepackage{comment}
\usepackage{caption}
\usepackage{subfigure}
\usepackage{fixltx2e}
\usepackage{changepage}
\usepackage{listings}
\DeclareUnicodeCharacter{00A0}{ }
\pagestyle{fancy}

% http://tex.stackexchange.com/questions/33519/vertical-line-in-matrix-using-latexit
% Fix for bmatrix
\makeatletter
\renewcommand*\env@matrix[1][*\c@MaxMatrixCols c]{%
  \hskip -\arraycolsep
  \let\@ifnextchar\new@ifnextchar
  \array{#1}}
\makeatother

% My commands
\newcommand{\equ}[1]{\begin{align}#1\end{align}}
\newcommand{\tb}[1]{\textbf{#1}\\}
\newcommand{\tsub}[1]{\textsubscript{#1}}
\newcommand{\tsup}[1]{\textsuperscript{#1}}

% End of my commands

\begin{document}
\section{Opgave}

\subsection{a)}
\tb{}

Konverter følgende fem decimaltal til hexadecimaltal: 0, 13, 314, 1337, 7913. Giv alle svarene på tre cifre.




\subsection{b)}
\tb{}

Konverter følgende hexadecimaltal til decimaltal: 800, F2F, FFF, 10, DAD. 

(Tallene er i base 10, med mindre andet er angivet.)
$$ 800_{16} = 8*256 + 0 * 16 + 0 * 1 = 2048_{10}$$
$$ F2F_{16} = 15 * 256 + 2 * 16 + 15 * 1 = 3887_{10}$$
$$ FFF_{16} = 15 * 256 + 15 * 16 + 15 * 1 = 4095_{10}$$
$$ 10_{16} = 0* 256 + 1 * 16 + 0 * 1 = 16_{10}$$
$$ DAD_{16} = 14 * 256 + 10 * 16 + 14 * 1 = 3758_{10}$$

\subsection{c)}
\tb{}

\begin{center}
  \begin{tabular}{| c | c | c | c |}
    \hline
     \textbf{4096} & \textbf{256} & \textbf{16} & \textbf{1}\\ \hline
    1 & e & e & 9\\
    \hline
  \end{tabular}
\end{center}

Da det er nemmere at omregne hexadecimal til binæretal, bruger vi tallene fra opgave 1.1\\
Vi opstiller følgende tabel, så vi nemt og hurtigt kan omregne til binære tal. Vi starter med at ignorer fortegnet på talene.
\begin{center}
  \begin{tabular}{ | l | c | c | c |  c |  c |  c |  c |  c |  c |  c |  c |  c |  c | }
    \hline
    & 1 & 2 & 4 & 8 & 16 & 32 & 64 & 128 & 256 & 512 & 1024 & 2048 & Sign\\ \hline
    $0_{16}$ &        0 & 0 & 0 & 0 & 0 & 0 & 0 & 0 & 0 & 0 & 0 & 0 & 0 \\ \hline
    $13_{16}$ &      1 & 0 & 1 & 1 & 0 & 0 & 0 & 0 & 0 & 0 & 0 & 0 & 0 \\ \hline
    $314_{16}$ &    0 & 1 &0 & 1 & 1 & 1 & 0 & 0 & 1 & 0 & 0 & 0 & 0 \\ \hline
    $539_{16}$ &    1 & 0 & 0 & 1 & 1 & 1 & 0 & 0 &1 & 0 & 1 & 0 & 0 \\ \hline
    $1ee9_{16}$ & 1 & 0 & 0 & 1 & 0 & 1 & 1 & 1 & 0 & 1 & 1 & 1 & 1 \\
    \hline
  \end{tabular}
\end{center}

Vi ser, at  $1ee9_{16}$ er for stort til at blive repræsenteret med 12 bit. Vi har derfor et overflow, når vi omregner  $1ee9_{16}$.\\

Vi ser bort fra overflowet og fortsætter med at konvertere til 12 bit 2-komplement
\begin{center}
  \begin{tabular}{ | l | c | c | c |  c |  c |  c |  c |  c |  c |  c |  c |  c |  c | }
    \hline
    & 1 & 2 & 4 & 8 & 16 & 32 & 64 & 128 & 256 & 512 & 1024 & 2048 & Sign\\ \hline
    $0_{16}$ &        0 & 0 & 0 & 0 & 0 & 0 & 0 & 0 & 0 & 0 & 0 & 0 & 0 \\ \hline
    $-13_{16}$ &      0 & 1 & 0 & 0 & 1 & 1 & 1 & 1 & 1 & 1 & 1 & 1 & 1 \\ \hline
    $-314_{16}$ &    1 & 0 & 0 & 0 & 0 & 0 & 1 & 1 & 1 & 1 & 1 & 1 & 1 \\ \hline
    $-539_{16}$ &    0 & 1 & 1 & 0 & 0 & 0 & 1 & 1 & 0 & 1 & 0 & 1 & 1 \\ \hline
    $-1ee9_{16}$ &   0 & 1 & 1 & 0 & 1 & 0 & 0 & 0 & 1 & 0 & 0 & 0 & 1 \\
    \hline
  \end{tabular}
\end{center}



















\end{document}